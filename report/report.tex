\documentclass[11pt]{article}

\usepackage[a4paper, total={16cm, 24cm}]{geometry}
\usepackage[portuguese]{babel}
\usepackage[utf8]{inputenc}
\usepackage{graphicx}
\usepackage{amsmath}
\usepackage{tikz}
    \usetikzlibrary{shadows}
\usepackage{booktabs}
\usepackage[colorlinks=true]{hyperref}
\usepackage{listings}
    \renewcommand\lstlistingname{Listagem}
    \lstset{numbers=left, numberstyle=\tiny, numbersep=5pt, basicstyle=\footnotesize\ttfamily, frame=tb,rulesepcolor=\color{gray}, breaklines=true}
\usepackage{blindtext}

% -------------------------------------------------------------------------------------------
\title
{
    \includegraphics[width=0.4\textwidth]{imgs/university.png}
    \\[0.1cm]
    \textbf{2º Trabalho} \\
    Inteligência Artificial
}

\author
{
    \textbf{Professora:} Irene Rodrigues \\
    \textbf{Realizado por:} Filipe Alfaiate (43315), Miguel de Carvalho (43108), João Pereira (42864) 
}
\date{\today}

% -------------------------------------------------------------------------------------------
%                                Body                                                       %
% -------------------------------------------------------------------------------------------

\begin{document}
\maketitle

% -------------------------------------------------------------------------------------------
1 - a) O espaço de estados e os operadores de transição de estados encontra-se no ficheiro

\hspace{1,1cm}\verb|mesa.pl|. Para calcular o estado inicial atribuimos o dominio (pessoas)

\hspace{1,1cm}a cada variável (cadeiras).

\hspace{0,6cm}b) Utilizar os seguintes comandos \verb|[pni].| e em seguida 
\verb|pesquisa(agente,largura).|

\hspace{0,6cm}c)

\hspace{1cm}i. O número de estados visitados pelo algoritmo das linhas anteriores é \textbf{64}.

\hspace{1cm}ii. O número de estados visitados que estão simultaneamente em memória é \textbf{36}.

\hspace{0,6cm}d) As duas heurísticas admissíveis para estimar o custo de um estado até à solução

\hspace{1,1cm}são \textbf{Heurística Greedy} (g) e a \textbf{Heurística A*} (a).

\hspace{0,6cm}e)

\hspace{1cm}i. 1 - Manuel

\hspace{1,4cm}2 - Joaquim

\hspace{1,4cm}3 - Madalena

\hspace{1,4cm}4 - Maria
\newline

\hspace{1cm}ii. 1 - Manuel

\hspace{1,5cm}2 - Joaquim

\hspace{1,5cm}3 - Madalena

\hspace{1,5cm}4 - Maria

\hspace{1,5cm}5 - Ana

\hspace{1,5cm}6 - Júlio
\newline

\hspace{1cm}iii. 1 - Matilde

\hspace{1,6cm}2 - Joaquim

\hspace{1,6cm}3 - Gabriel

\hspace{1,6cm}4 - Maria

\hspace{1,6cm}5 - Manuel

\hspace{1,6cm}6 - Madalena

\hspace{1,6cm}7 - Ana

\hspace{1,6cm}8 - Júlio
\newpage
\hspace{1cm}iiii. 1 - Manuel

\hspace{1,7cm}2 - Joaquim

\hspace{1,7cm}3 - Matilde

\hspace{1,7cm}4 - Madalena

\hspace{1,7cm}5 - Ana

\hspace{1,7cm}6 - Julio

\hspace{1,7cm}7 - Gabriel

\hspace{1,7cm}8 - Filipe

\hspace{1,7cm}9 - Miguel

\hspace{1,7cm}10 - Joao

\hspace{1,7cm}11 - CAP

\hspace{1,7cm}12 - Maria

% -------------------------------------------------------------------------------------------
\end{document}