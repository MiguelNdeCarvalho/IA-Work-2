\documentclass[11pt]{article}

\usepackage[a4paper, total={16cm, 24cm}]{geometry}
\usepackage[portuguese]{babel}
\usepackage[utf8]{inputenc}
\usepackage{graphicx}
\usepackage{amsmath}
\usepackage{tikz}
    \usetikzlibrary{shadows}
\usepackage{booktabs}
\usepackage[colorlinks=true]{hyperref}
\usepackage{listings}
    \renewcommand\lstlistingname{Listagem}
    \lstset{numbers=left, numberstyle=\tiny, numbersep=5pt, basicstyle=\footnotesize\ttfamily, frame=tb,rulesepcolor=\color{gray}, breaklines=true}
\usepackage{blindtext}

% -------------------------------------------------------------------------------------------
\title
{
    \includegraphics[width=0.4\textwidth]{imgs/university.png}
    \\[0.1cm]
    \textbf{2º Trabalho} \\
    Inteligência Artificial
}

\author
{
    \textbf{Professora:} Irene Rodrigues \\
    \textbf{Realizado por:} Filipe Alfaiate (43315), Miguel de Carvalho (43108), João Pereira (42864) 
}
\date{\today}

% -------------------------------------------------------------------------------------------
%                                Body                                                       %
% -------------------------------------------------------------------------------------------

\begin{document}
\maketitle

% -------------------------------------------------------------------------------------------
1 - a) O espaço de estados e os operadores de transição de estados encontra-se no ficheiro

\hspace{1,1cm}\verb|agente.pl|.

\hspace{0,6cm}b) Utilizar os seguintes comandos \verb|[pni].| e em seguida 
\verb|pesquisa(agente,largura).|

\hspace{0,6cm}c)

\hspace{1cm}i. O número de estados visitados pelo algoritmo das linhas anteriores é \textbf{64}.

\hspace{1cm}ii. O número de estados visitados que estão simultaneamente em memória é \textbf{36}.

\hspace{0,6cm}d) As duas heurísticas admissíveis para estimar o custo de um estado até à solução

\hspace{1,1cm}são \textbf{Heurística Greedy} (g) e a \textbf{Heurística A*} (a).

\hspace{0,6cm}e) Utilizar os seguintes comandos \verb|[pi].| e em seguida 
\verb|pesquisa(agente,g).|

\hspace{0,6cm}f)

\hspace{1cm}i. O número de estados visitados pelo algoritmo das linhas anteriores é \textbf{10}.

\hspace{1cm}ii. O número de estados visitados que estão simultaneamente em memória é \textbf{11}.


\begin{table}[h!]
    \begin{center}
        \begin{tabular}{l|c|c|c|c}
            \textbf{Algoritmo} & \textbf{Nº nós visitados} & \textbf{Nº nós em memória} & \textbf{Encontra solução} & \textbf{Melhor solução}\\
            \hline
            largura      & 64        & 36 & sim & sim \\
            \hline
            profundidade & 20        & 15 & sim & não \\
            \hline
            iterativo    & 3968322   & 18 & sim & não \\
            \hline
            greedy       & 10        & 11  & sim & sim \\
            \hline
            a*           & 44        & 22 & sim & não
        \end{tabular}
        \caption{Resumo do problema do Agente.}
        \label{tab:table1}
    \end{center}
\end{table}

% -------------------------------------------------------------------------------------------
\newpage

2 - a) O espaço de estados e os operadores de transição de estados encontra-se no ficheiro

\hspace{1,1cm}\verb|caixa.pl|.

\hspace{0,6cm}b) Utilizar os seguintes comandos \verb|[pni].| e em seguida 
\verb|pesquisa(caixa,largura).|

\hspace{0,6cm}c)

\hspace{1cm}i. O número de estados visitados pelo algoritmo das linhas anteriores é \textbf{4211}.

\hspace{1cm}ii. O número de estados visitados que estão simultaneamente em memória é \textbf{606}.

\hspace{0,6cm}d) As duas heurísticas admissíveis para estimar o custo de um estado até à solução

\hspace{1,1cm}são \textbf{Heurística Greedy} (g) e a \textbf{Heurística A*} (a).

\hspace{0,6cm}e) Utilizar os seguintes comandos \verb|[pi].| e em seguida 
\verb|pesquisa(caixa,g).|

\hspace{0,6cm}f)

\hspace{1cm}i. O número de estados visitados pelo algoritmo das linhas anteriores é \textbf{341}.

\hspace{1cm}ii. O número de estados visitados que estão simultaneamente em memória é \textbf{46}.


\begin{table}[h!]
    \begin{center}
        \begin{tabular}{l|c|c|c|c}
            \textbf{Algoritmo} & \textbf{Nº nós visitados} & \textbf{Nº nós em memória} & \textbf{Encontra solução} & \textbf{Melhor solução}\\
            \hline
            largura      & 4211 & 606 & sim & sim \\
            \hline
            profundidade & 1989 & 180 & sim & não \\
            \hline
            iterativo    & -    & -   & não & -   \\
            \hline
            greedy       & 341  & 46  & sim & sim \\
            \hline
            a*           & 2078 & 525 & sim & não
        \end{tabular}
        \caption{Resumo do problema da Caixa.}
        \label{tab:table1}
    \end{center}
\end{table}

% -------------------------------------------------------------------------------------------
\end{document}