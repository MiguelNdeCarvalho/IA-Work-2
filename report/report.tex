\documentclass[11pt]{article}

\usepackage[a4paper, total={16cm, 24cm}]{geometry}
\usepackage[portuguese]{babel}
\usepackage[utf8]{inputenc}
\usepackage{graphicx}
\usepackage{amsmath}
\usepackage{tikz}
    \usetikzlibrary{shadows}
\usepackage{booktabs}
\usepackage[colorlinks=true]{hyperref}
\usepackage{listings}
    \renewcommand\lstlistingname{Listagem}
    \lstset{numbers=left, numberstyle=\tiny, numbersep=5pt, basicstyle=\footnotesize\ttfamily, frame=tb,rulesepcolor=\color{gray}, breaklines=true}
\usepackage{blindtext}

% -------------------------------------------------------------------------------------------
\title
{
    \includegraphics[width=0.4\textwidth]{imgs/university.png}
    \\[0.1cm]
    \textbf{2º Trabalho} \\
    Inteligência Artificial
}

\author
{
    \textbf{Professora:} Irene Rodrigues \\
    \textbf{Realizado por:} Filipe Alfaiate (43315), Miguel de Carvalho (43108), João Pereira (42864) 
}
\date{\today}

% -------------------------------------------------------------------------------------------
%                                Body                                                       %
% -------------------------------------------------------------------------------------------

\begin{document}
\maketitle

% -------------------------------------------------------------------------------------------
\section{}

\hspace{0,6cm}a) O espaço de estados e os operadores de transição de estados encontra-se no ficheiro.
\verb|mesa.pl|. Para calcular o estado inicial atribuimos o dominio (pessoas) a cada variável (cadeiras).

b) Utilizar o seguinte comando '\verb|[mesa].|' e em seguida '\verb|p.|'.

c) Utilizar os seguinte comando '\verb|[mesa].|' e em seguida '\verb|p.|'. Para correr de forma desejada 
é necessário comentar o predicado \verb|back(e([], A), A).| que se encontra nas linhas 19-21 e descomentar
o predicado \verb|back(E, Sol).| que se encontra nas linhas 24-28.

d) Não encontrámos uma forma diferente que apresente uma melhoria significativa que diminua a complexidade
(temporal e epacial).

e)

\hspace{0,7cm}i. 1 - Manuel

\hspace{1,1cm}2 - Joaquim

\hspace{1,1cm}3 - Madalena

\hspace{1,1cm}4 - Maria
\newline

\hspace{0,6cm}ii. 1 - Manuel

\hspace{1,1cm}2 - Joaquim

\hspace{1,1cm}3 - Madalena

\hspace{1,1cm}4 - Maria

\hspace{1,1cm}5 - Ana

\hspace{1,1cm}6 - Júlio
\newline

\hspace{0,5cm}iii. 1 - Matilde

\hspace{1,1cm}2 - Joaquim

\hspace{1,1cm}3 - Gabriel

\hspace{1,1cm}4 - Maria

\hspace{1,1cm}5 - Manuel

\hspace{1,1cm}6 - Madalena

\hspace{1,1cm}7 - Ana

\hspace{1,1cm}8 - Júlio
\newpage
\hspace{0,5cm}iv. 1 - Manuel

\hspace{1,1cm}2 - Joaquim

\hspace{1,1cm}3 - Matilde

\hspace{1,1cm}4 - Madalena

\hspace{1,1cm}5 - Ana

\hspace{1,1cm}6 - Julio

\hspace{1,1cm}7 - Gabriel

\hspace{1,1cm}8 - Filipe

\hspace{1,1cm}9 - Miguel

\hspace{1,1cm}10 - Joao

\hspace{1,1cm}11 - Inácio

\hspace{1,1cm}12 - Maria
\newline
% -------------------------------------------------------------------------------------------
\section{}

\hspace{0,6cm}a) O espaço de estados e os operadores de transição de estados encontra-se no ficheiro
\verb|sudoku.pl|. Para calcular o estado inicial atribuimos o dominio (1-9) a cada variável (posição).

b) Utilizar os seguintes comandos '\verb|[sudoku].|' e em seguida '\verb|p.|'.

c) Utilizar os seguinte comando '\verb|[sudoku].|' e em seguida '\verb|p.|'. Para correr de forma desejada 
é necessário comentar o predicado \verb|back(e([], A), A).| que se encontra nas linhas 19-21 e descomentar
o predicado \verb|back(E, Sol).| que se encontra nas linhas 24-28.

d) Não encontrámos uma forma diferente que apresente uma melhoria significativa que diminua a
complexidade (temporal e epacial).
\newline

Exemplo de um output:
\begin{table}[h!]
    \begin{center}
        \begin{tabular}{|c|c|c|c|c|c|c|c|c|}
            \hline
            5 & 1 & 9 & 4 & 2 & 8 & 6 & 7 & 3 \\
            \hline
            6 & 3 & 4 & 5 & 7 & 9 & 1 & 8 & 2 \\
            \hline
            7 & 2 & 8 & 3 & 1 & 6 & 9 & 5 & 4 \\
            \hline
            3 & 5 & 2 & 1 & 8 & 4 & 7 & 9 & 6 \\
            \hline
            9 & 7 & 6 & 2 & 3 & 5 & 4 & 1 & 8 \\
            \hline
            8 & 4 & 1 & 9 & 6 & 7 & 3 & 2 & 5 \\
            \hline
            4 & 9 & 3 & 7 & 5 & 2 & 8 & 6 & 1 \\
            \hline
            2 & 6 & 7 & 8 & 4 & 1 & 5 & 3 & 9 \\
            \hline
            1 & 8 & 5 & 6 & 9 & 3 & 2 & 4 & 7 \\
            \hline
        \end{tabular}
    \end{center}
\end{table}
% -------------------------------------------------------------------------------------------
\end{document}